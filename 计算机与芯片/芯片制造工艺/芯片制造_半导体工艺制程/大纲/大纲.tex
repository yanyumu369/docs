\documentclass{standalone}
\usepackage[UTF8]{ctex} % 中文字体支持
% \usepackage{geometry}
% \geometry{height=2800mm, width=200mm}
\usepackage{amsmath, amssymb}
\usepackage{tikz, xcolor}
\usetikzlibrary{shapes, arrows}
% \definecolor{mybrown}{RGB}{214,91,9}

\usepackage[edges]{forest}

\usepackage{sansmath}
% \renewcommand{\familydefault}{\sfdefault}
% \renewcommand{\sfdefault}{phv}

% \tikzset{parent/.style={text width=0.6cm,rounded corners=2pt},
%     child/.style={text width=0.5cm,rounded corners=2pt},
%     grandchild/.style={},
% }


\begin{document}
\begin{forest}
    for tree={
    grow'=0, folder, draw,
    % l sep=0.6cm,
    % s sep=0.4cm,
    % minimum height=0.5cm,
    % minimum width=0.5cm,
    % draw, anchor=parent first, % 左对齐
    align=left,
    % where n children=0{tier=word}{}, % 叶子结点对齐
    },
    [芯片制造—半导体工艺制程
    [氧化,
    [作用:在硅表面上通过热氧化形成二氧化硅(可能是硅在形成半导体器件方面最有用的)]
    [二氧化硅,
    [用途:表面钝化层(保护)、掺杂阻挡层、表面介电层(场氧化层)、\\器件介电层(MOS栅极/电容器)]
    ]
    [热氧化机制,
    [反应:{$Si+O_2 \rightarrow SiO_2$}]
    [线性阶段和抛物线阶段(1000$\sim$1200-Å)]
    [用水蒸气代替氧气可以实现更快氧化,原因是{$OH^-$}(羟基离子)通过\\晶圆上已经存在的氧化层的扩散速度更快]
    [氧化速率的影响
    [晶圆取向:晶面上硅原子数越多,氧化速率越快]
    [掺杂剂浓度:高掺杂区比低掺杂区氧化更快]
    [氧化杂质:氧化气氛中的杂质,特别是来自盐酸(HCl)的氯]
    [多晶硅氧化:相比单晶硅,多晶硅的速率可能更快、更慢或类似]
    [差异氧化和氧化台阶]
    ]
    [热氧化方法
    [管式炉:水平管式炉和垂直管式炉]
    [快速热氧化:晶圆表面辐射加热,热源包括石墨加热器、微波、\\等离子弧、卤钨灯,升温速率每秒50-100°C]
    [高压氧化:10-25个大气压,目的是降低氧化温度,减少位错的增长\\或者保持工艺温度,缩短氧化时间]
    ]
    [氧化剂源
    [干氧气:生长MOS栅氧化层($\sim$1000-Å)的首选方法]
    [水蒸气]
    [起泡器]
    [干氧化(dryox):气态氢气和氧气直接引入氧化炉,\\气体混合并在高温下形成蒸气]
    [掺氯氧化:当将氯加入氧化物中时,清洁度和器件性能得到改善。\\氯倾向于减少氧化物层中的移动离子电荷,减少氧化物和\\硅表面的结构缺陷,并减少氧化物-硅界面的电荷。]
    ]
    [氧化工艺:预清洗-蚀刻-氧化,实际的氧化在不同的气体循环中进行]
    ]
    ]
    [图形化工艺,
    [曝光]
    [显影]
    ]
    [掺杂,
    [目的:形成N-P或N-P结(结的确切位置是电子的浓度与空穴的浓度相等的位置)]
    [热扩散
    [原理:掺杂材料通常通过顶部二氧化硅层中的一个通孔被引入到晶圆的裸露顶面。\\随着加热,它们散布到晶圆的大部分空间]
    [沉积
    [将掺杂物引入晶圆表面]
    [移动机制:空位和间隙]
    [影响沉积的因素:扩散率(掺杂物在晶圆材料中的移动速率,随温度增加而增加)\\和最大固溶度(掺杂物在晶圆中的最大浓度)]
    [影响器件性能的因素:掺杂物的表面浓度和扩散到晶圆的原子总数]
    ]
    [推进氧化
    [将掺杂物推进(散布)到所需深度]
    [目的:掺杂物重新分布到晶圆的更深处(高斯分布)和\\氧化暴露的硅表面(氧气或水蒸气)]
    [掺杂后氧化的影响:N型-堆积,P型-耗尽(掺杂物在\\硅和二氧化硅中溶解度的不同)]
    ]
    [挑战:横向扩散、超浅结、不佳的掺杂控制、表面污染干扰和位错的产生]
    ]
    [离子注入
    [原理:掺杂材料被直接射入晶圆表面,大部分掺杂原子停留在表面以下]
    [MOS晶体管新的掺杂要求:低掺杂浓度控制和超浅结]
    [优点
    [没有横向扩散,接近室温,掺杂原子位于晶圆表面下方,广泛的掺杂浓度]
    [离子注入可以更好地控制晶圆中掺杂物的位置和数量]
    [光刻胶和薄金属层可以与通常的二氧化硅层一起用作掺杂阻挡层]
    ]
    [离子注入机
    [离子室、质量分析仪、加速管、束流聚焦、\\中性束流陷阱、束流扫描、离子注入掩模]
    ]
    [离子注入区域的\\杂质浓度
    [注入的原子数(剂量)
    [束流密度(每平方厘米离子数)]
    [注入物]
    ]
    [离子在晶圆中的位置
    [离子进入能量]
    [晶圆取向]
    [离子减速和停止机制]
    ]
    [离子减速和停止机制
    [正离子与晶体中带负电荷的电子的相互作用]
    [与晶圆原子核的物理碰撞]
    ]
    [晶体损伤
    [类型:晶格损伤、损伤簇、空位-间隙]
    [退火:恢复晶体损伤]
    ]
    [沟道效应
    [当离子束与晶圆主轴平行时,离子可以沿\\沟道向下传播,达到计算深度的10倍]
    [解决方案:阻塞非晶表面层、晶圆定向扭转、\\在晶圆表面形成损伤层]
    ]
    ]
    [注入层评估
    [四点探针和范德堡法:量薄膜电阻]
    [散布电阻技术和电容电压技术:确定浓度分布、剂量和结深]
    [光学剂量测定:用剂量仪测量光刻胶薄膜在离子注入前后的吸收,\\并绘制表面轮廓]
    ]
    [用途
    [预沉积倒掺杂阱:使用高能注入物创建深P型阱]
    [超浅结:需要能量较低的离子注入工艺,以最小化表面损伤和沟道效应]
    [MOS栅阈值电压调整:阈值电压对栅下晶圆表面的掺杂浓度非常敏感]
    ]
    [掺杂前景展望
    [缺点
    [设备昂贵且复杂]
    [培训和维护时间更长,要求更严格]
    [使用高电压和更多有毒气体]
    ]
    [最大担忧:退火工艺完全消除注入物引起的损伤的能力]
    [技术替代:等离子体浸没(类似于离子研磨和溅射)]
    ]
    ]
    ]
    [薄膜沉积,
    [引言
    [薄膜材料:半导体、电介质(绝缘体)和导体]
    [沉积技术:化学气相沉积(CVD)、物理气相沉积(PVD)、电镀、旋涂和蒸发]
    [不同层的作用:称为外延层的沉积掺杂硅层、金属间介电层,垂直沟槽电容器、\\金属间导电塞、金属导电层、最终表面钝化层]
    ]
    [薄膜参数
    [厚度均匀,连续且无针孔,以防止污染物通过或者夹层短路]
    [挑战:台阶处的过度减薄,填充高深宽比沟槽(沟槽边缘和底部的减薄)]
    ]
    [化学气相沉积,
    [原理:化学物质在沉积室中混合并反应形成蒸气,然后\\原子或分子沉积在晶圆表面并形成薄膜]
    [化学反应类型:热解、还原、氧化和氮化]
    [生长阶段:成核-核生长成岛屿-岛屿扩展形成连续薄膜]
    [CVD系统
    [基本组成:源柜、反应室、能量源、晶圆夹持器(舟)\\以及装载和卸载机构]
    [能量源:热传导、对流、感应射频、辐射、等离子体或紫外线]
    [主要类型:常压(AP)和低压(LP);冷壁和热壁]
    [特殊系统:气相外延、金属有机CVD、非CVD分子束外延]
    ]
    [常压CVD系统
    [水平射频感应加热APCVD:利用射频波与晶圆的石墨夹持器\\的分子“耦合”,使石墨升温]
    [桶式辐射感应加热APCVD:内表面周围装有高强度石英加热器,\\晶圆放在石墨支架上]
    [饼式传导加热APCVD]
    [连续传导加热APCVD:让晶圆夹持器在一系列喷嘴下来回移动\\或者让晶圆在一个分配气体的羽化器下的皮带上移动]
    [水平传导加热APCVD]
    ]
    [低压CVD
    [降低腔室中的压力可增加分子的平均自由程和薄膜均匀性;\\降低压力也可以降低沉积温度]
    [水平传导对流\\加热LPCVD
    [生长典型厚度均匀度为 ±5\% 的多晶硅、\\二氧化硅和氮化硅薄膜]
    [主要的沉积变量:温度、压力、气体流量、\\气体分压和晶圆间距]
    ]
    [超高真空CVD
    [降低压力可以保持低沉积温度,从而有助于减少晶体损伤,\\降低热预算,进而减少掺杂区域的横向扩散]
    [在超低真空条件下沉积硅和硅锗(SiGe)]
    [过程:一开始将压力降至$1-5\times 10^{-9}$-mbar,\\然后升高至$10^{-3}$-mbar(沉积压力)]
    ]
    [等离子体增强型CVD\\(PECVD)
    [用氮化硅代替二氧化硅钝化层\\导致了PECVD技术的发展]
    [等离子体可以增强沉积能量,\\从而降低沉积温度]
    ]
    [高密度等离子体CVD
    [沉积和原位蚀刻:第一次沉积后底部通常变薄,蚀刻掉\\肩部并重新沉积,可形成均匀的层厚和更平坦的表面]
    [在含有氧气和硅烷的CVD腔室中形成等离子体场,\\用于沉积二氧化硅]
    [被等离子体激发并被引导到晶圆表面的氩通过\\溅射作用将材料从表面和沟槽中去除]
    ]
    ]
    ]
    [原子层沉积(ALD)
    [在ALD中,前驱体按顺序进入腔室,由吹扫气体分离]
    [每个薄膜阶段都以单层速率生长,因此控制非常精确]
    [ALD将层厚从通常的CVD水平300-Å降至12-Å]
    [用途:极薄的二氧化硅栅极、用氧化铝等材料填充深沟槽,\\以及为铜金属化工艺创建阻挡金属层]
    ]
    [气相外延(VPE):不同于所描述的CVD系统,它能够沉积化合物材料,例如砷化镓]
    [分子束外延(MBE)
    [MBE是一种蒸发工艺,而不是CVD工艺]
    [沉积室保持在$10^{-10}$-torr中]
    [射流单元:包含晶圆上所需目标材料的高纯样品]
    [电子束被引导到目标材料的中心,并将其加热到液态,原子从材料中蒸发出来]
    [通常的硅掺杂源在MBE系统中不可用。\\固体镓用于P型掺杂,锑用于N型掺杂。\\MBE几乎不可能进行磷沉积]
    [主要优点;低温、在一次工艺步骤中生长多层膜、薄膜生长率低(也是缺点)]
    [可以将薄膜生长和质量分析仪器整合到腔室内]
    [MBE已被用于制造特殊微波器件和砷化镓等化合物半导体]
    ]
    [金属有机物CVD(MOCVD)
    [使用两种化学物质:卤化物和金属有机物]
    [与MBE不同,MOCVD可以像InGaAsP器件一样沉积磷]
    [MOCVD工艺制造的常见器件有光电阴极、高功率LED、长波激光器、\\可见激光和橙色LED]
    [外延III-V半导体层,包括GaAs、AlAs、AlGaAs,InGaAs和InP]
    [III–V半导体的Δ掺杂、量子点、量子线、量子阱的生长、\\调制掺杂异质结构和选择性区域外延生长]
    ]
    [外延硅
    [硅化学源
    [四氯化硅:首选的硅化学源;可逆反应]
    [硅烷:热解反应;更多用作多晶硅和二氧化硅沉积]
    [二氯硅烷:低温硅源]
    ]
    ]
    [多晶硅和非晶硅沉积
    [用途:MOS栅、SRAM器件中的负载电阻、沟槽填充、EEPROM中的多层多晶硅、\\接触阻挡层、双极器件中的发射极,以及硅化物金属化方案的一部分]
    [沉积条件:600至650°C范围]
    [化学源:100\%硅烷或含有$N_2$或$H_2$的气流;气体掺杂源(原位掺杂)]
    ]
    [绝缘体和电介质
    [二氧化硅
    [用途:最终表面钝化层、多金属化方案中的介电层、多晶硅层和金属化层之间的绝缘层、\\掺杂阻挡层、扩散源和隔离区、硅栅结构]
    [化学反应:$SiH_4+O_2$、$SiCl_2H_2+2NO_2$、正硅酸乙酯(TEOS)热解、$SiH_4+4N_2O$]
    [掺杂的二氧化硅
    [掺磷:磷化氢($PH_3$);防潮、吸收离子污染物、增加流动性]
    [掺硼:二硼烷($B_2H_6$);改善流动性]
    ]
    ]
    [氮化硅
    [用途:二氧化硅的替代品、更硬、更好的防潮层和防纳层、更高的介电强度、抗氧化]
    [化学反应:$3SiH_4+4NH_3$、$3SiCl_2H_2+4NH_3$]
    ]
    [介电常数$k$
    [高$k$介质:电容器、栅极介质]
        [低$k$介质:介质阻挡层]
    ]
    ]
    ]
    [金属化,
    [淀积方法
    [溅射法:铝合金和其他金属]
    [低压CVD法:多晶硅、钨和其他难熔金属]
    [用电镀金属化技术的双大马士革铜工艺:铜]
    ]
    ]
    [其它,
    [热处理
    [合金化:确保金属和晶圆表面之间的良好电接触。]
    ]
    ]
    ]
    \draw (-5mm, -1400mm) rectangle (205mm, 1400mm);
\end{forest}
\end{document}
