\documentclass[margin=5pt]{standalone}
\usepackage{tikz}
\usetikzlibrary{calc}

\begin{document}

\newcommand{\tikzAngleOfLine}{\tikz@AngleOfLine}
  \def\tikz@AngleOfLine(#1)(#2)#3{%
  \pgfmathanglebetweenpoints{%
    \pgfpointanchor{#1}{center}}{%
    \pgfpointanchor{#2}{center}}
  \pgfmathsetmacro{#3}{\pgfmathresult}%
  }

  \begin{tikzpicture}
    \coordinate (A) at (1,1);
    \coordinate (B) at ($(A)+(25:3)$);
    \coordinate (C) at ($(A)+(100:5)$);
    \draw (A) node[left]{$A$} -- (B) node[right]{$B$}node[midway,below]{$c$} -- (C)node[above]{$C$}node[midway,above]{$a$} -- (A)node[midway,left]{$b$};

    \tikzAngleOfLine(A)(B){\AngleStart}
    \tikzAngleOfLine(A)(C){\AngleEnd}
    \draw[red,<->] (A)+(\AngleStart:2cm) arc (\AngleStart:\AngleEnd:2 cm);
    \node[circle,fill=green] at ($(A)+({(\AngleStart+\AngleEnd)/2}:1 cm)$) {$\alpha$};
\end{tikzpicture}
\end{document} 